\documentclass[oneside]{ctexbook}
\usepackage{xeCJK}
\usepackage{tabularx}%表格包
\usepackage{booktabs}%三线表包
\usepackage{amsmath}%数学公式包
\usepackage{fancyhdr}%设置页脚页眉需要的包
\usepackage{listings}%代码环境包
\usepackage{authblk}%多个作者包
\usepackage{fontspec}%设置字体包
\usepackage{color,xcolor}%颜色包
\pagestyle{fancy}
\author{钱昌发} 
\author{王淇}
\author{周致远}
\affil{安徽大学}
\title{数学模型参考答案及代码}
%这里是页眉
\lhead{安徽大学}%左边
%\chead{中间}%中间
\rhead{数学模型答案}%右边
%这里是页脚
%\lfoot{左边}%页脚左边
\cfoot{第\thepage 页}%页脚中间
%\rfoot{右边}%页脚右边
\renewcommand{\headrulewidth}{0.4pt}%页眉
\renewcommand{\headwidth}{\textwidth}
\renewcommand{\footrulewidth}{0.4pt}
\renewcommand\Authands{,}
\newfontfamily\con{Consolas}
\definecolor{mygreen}{rgb}{0,0.6,0}
\lstset{
	basicstyle=\fontsize{9pt}{10}\con, % size of fonts used for the code或改成\small\monaco稍大
	numbers=left,                        % 设置行号
	numberstyle=\tiny,            % 设置行号字体大小
	columns=fullflexible,
	breaklines=true,                 % automatic line breaking only at whitespace
	captionpos=b,                    % sets the caption-position to bottom
	tabsize=4,
	commentstyle=\color{mygreen},    % 设置注释颜色
	keywordstyle=\color{blue},       % 设置keyword颜色
	stringstyle=\color,     % string literal style
	%frame=single,                        % 设置有边框
	language=matlab,
}
\begin{document}
	\maketitle
	\newpage
	\noindent 第一题答案:
\begin{lstlisting}
  clc;clear all;close all;
  f=-[50 100];
  A=[1 1
     2 1
     0 1];
  b=[300 400 250];
  lb=[0 0];
  ub=[inf inf];
  a=linprog(f,A,b,[],[],lb,ub)
  %answer:
  Optimization terminated.
  a =
  50.0000
  250.0000
\end{lstlisting}

\noindent 第二题答案:
\begin{lstlisting}
(1)
  f=[1 2 3];
  A=[-2 1 -1
      1 1 2
     0 -1 1];
  b=[-4 8 -2];
  lb=[0 0 0 ];
  ub=[inf inf inf];
  a=linprog(f,A,b,[],[],lb,ub)
  %answer:
  Optimization terminated.

  a =

  3.0000
  2.0000
  0.0000
(2)
  f=-[5 2 4];
  A=[-3 -1 -2
  -6 -3 -5];
  b=[-2 -10];
  lb=[0 0 0];
  ub=[inf inf inf];
  a=linprog(f,A,b,[],[],lb,ub)
  answer:
  NO ANSWER

\end{lstlisting}
\noindent 第三题答案:\\
\textbf{解:}\par
可建立0-1规划模型,假设对于每个井位的钻与不钻两个状态设为变量$x_{i}$,其中变量的值为1时代表钻,值为0时代表不钻,由题意可得以下模型:
\begin{align}
&\notag \textnormal{min}z=\sum_{i=1}^{10}c_{i}x_{i}\\
&s.t.
\begin{cases}
x_{1}=x_{7}\\
0.5*(x_{1}+x_{7})+x_{5}=1\\
x_{3}+x_{5}=1\\
x_{4}+x_{5}=1\\
0\leq x_{i}\leq 1,\text{且}x_{1},x_{2},\dots,x_{10},\text{全部为整数}
\end{cases}
\end{align}
\noindent 第四题答案:
\begin{lstlisting}
  clear all; clc;close all;
  f=-[15 10 12 10 12 11 12 9 9 9 10 20 ...
  15 17 13 18 17 9 9 13 7 13 10 13 12];
  intcon=1:25;
  A=[];
  b=[];
  Aeq=[ones(1,5),zeros(1,20);
  zeros(1,5),ones(1,5),zeros(1,15);
  zeros(1,10),ones(1,5),zeros(1,10);
  zeros(1,15),ones(1,5),zeros(1,5);
  zeros(1,20),ones(1,5);
  full(sparse(ones(1,5),[1,6,11,16,21],ones(1,5))),zeros(1,4);
  full(sparse(ones(1,5),[2,7,12,17,22],ones(1,5))),zeros(1,3);
  full(sparse(ones(1,5),[3,8,13,18,23],ones(1,5))),zeros(1,2);
  full(sparse(ones(1,5),[4,9,14,19,24],ones(1,5))),zeros(1,1);
  full(sparse(ones(1,5),[5,10,15,20,25],ones(1,5)))];
  beq=ones(10,1);
  lb=zeros(25,1);
  ub=ones(25,1);
  [a,z]=intlinprog(f,intcon,A,b,Aeq,beq,lb,ub);
  reshape(a,[5,5])
  z
\end{lstlisting}
\noindent 第五题答案:
\begin{lstlisting}[language={[ANSI]C}]
  fun = @(x)50*x(1)+0.2*x(1)^2+50*x(2)+0.2*x(2)^2+...
  50*x(3)+0.2*x(3)^2+(x(1)-40)*4+(x(1)+x(2)-100)*4;
  x0=[1,1,1];
  A=[-1 0 0
      1 0 0
     -1 -1 0];
  b=[-40
     100
     -100];
  Aeq=[1,1,1];
  beq=[180];
  ub=[0 0 0];
  lb=[100 100 100];
  [x,fval]=fmincon(fun,x0,A,b,Aeq,beq,ub,lb)
  %answer:
  x =
      50.0000   60.0000   70.0001
\end{lstlisting}
\noindent 第六题答案:
\begin{lstlisting}[language={[ANSI]C}]
(1)
  %build a m file:
  function [c,ceq]=answer61(x)
  c=-(1-x(1))^3+x(2);
  ceq=[];
  %code:
  fun =@(x)-x(1);
  A = [];
  b = [];
  Aeq = [];
  beq = [];
  lb = [0,0];
  ub = [inf,inf];
  nonlcon = @answer61;
  x0 = [0,0];
  x = fmincon(fun,x0,A,b,Aeq,beq,lb,ub,nonlcon)
  %answer:
  x =
  0.9988    0.0000
(2)
  same as the last
(3)
  fun=@(x)(x(1)-3)^2+(x(2)-3)^2;
  x0=[0,0];
  A=[1,1];
  b=4;
  Aeq=[];
  beq=[];
  ub=[0,0];
  lb=[inf,inf];
  [x,fval]=fmincon(fun,x0,A,b,Aeq,beq,ub,lb)
  %answer:
  x =
  2.0000    2.0000
\end{lstlisting}
\noindent 第七题答案:
\begin{lstlisting}
  f=-[15,18,21,24;
  19,23,22,18;
  26,18,16,19;
  19,21,23,17;];
  intcon=1:16;
  A=[];
  b=[];
  Aeq=[ones(1,4),zeros(1,12);
  zeros(1,4),ones(1,4),zeros(1,8);
  zeros(1,8),ones(1,4),zeros(1,4);
  zeros(1,12),ones(1,4),
  full(sparse(ones(1,4),[1,5,9,13],ones(1,4))),zeros(1,3);
  full(sparse(ones(1,4),[2,6,10,14],ones(1,4))),zeros(1,2);
  full(sparse(ones(1,4),[3,7,11,15],ones(1,4))),zeros(1,1);
  full(sparse(ones(1,4),[4,8,12,16],ones(1,4)));];
  beq=ones(8,1);
  lb=zeros(16,1);
  ub=ones(16,1);
  [a,z]=intlinprog(f,intcon,A,b,Aeq,beq,lb,ub);
  reshape(a,[4,4])
  z
  %answer:
  ans =
  0     0     0     1
  0     1     0     0
  1     0     0     0
  0     0     1     0
  z =
  -96
\end{lstlisting}
\noindent 第八题:暂时空着\par
\noindent 第九题:\\
\textbf{解:}\par
设播放音乐节目的时间为$x_{1}$,播放新闻所用的时间为$x_{2}$,商业节目的时间为
$x_{3}$,依据题意建立线性规划模型:
\begin{align}
&\notag \textnormal{max}z=-17.5*x_{1}-40*x_{2}+250*x_{3},\\
&s.t.
\begin{cases}
x_{1}+x_{2}+x_{3}=12\\
0\leq x_{3}\leq 2.4\\
1\leq x_{2}\leq 12\\
0\leq x_{1}\leq 12
\end{cases}
\end{align}
然后发现这是一个很简单的线性规划模型,代码如下:\par
\begin{lstlisting}
  f=-[-17.5 -40 250];
  A=[];
  b=[];
  Aeq=[1 1 1];
  beq=12;
  lb=[0 1 0];
  ub=[12 12 2.4];
  [a,z]=linprog(f,A,b,Aeq,beq,lb,ub)
  %answer:
  Optimization terminated.
  a =
  8.6000
  1.0000
  2.4000
  z =
  -409.5000
\end{lstlisting}

\end{document}