\documentclass[oneside]{ctexbook}
\usepackage{titlesec}
\usepackage{xeCJK}
\usepackage{tabularx}
\usepackage{booktabs}
\usepackage{amsmath}
\usepackage{fancyhdr}%设置页脚页眉需要的包
\usepackage{listings}
\usepackage{authblk}
\usepackage{fontspec}
\usepackage{color,xcolor}
\pagestyle{fancy}
\author{钱昌发} 
\author{周致远}
\affil{安徽大学}
\title{数学模型参考答案及代码}
%这里是页眉
\lhead{安徽大学}%左边
%\chead{中间}%中间
\rhead{数学模型答案}%右边
%这里是页脚
%\lfoot{左边}%页脚左边
\cfoot{第\thepage 页}%页脚中间
%\rfoot{右边}%页脚右边
\renewcommand{\headrulewidth}{0.4pt}%页眉
\renewcommand{\headwidth}{\textwidth}
\renewcommand{\footrulewidth}{0.4pt}
\renewcommand\Authands{,}
\newfontfamily\con{Consolas}
\definecolor{mygreen}{rgb}{0,0.6,0}
\lstset{
	basicstyle=\fontsize{9pt}{10}\con, % size of fonts used for the code或改成\small\monaco稍大
	numbers=left,                        % 设置行号
	numberstyle=\tiny,            % 设置行号字体大小
	columns=fullflexible,
	breaklines=true,                 % automatic line breaking only at whitespace
	captionpos=b,                    % sets the caption-position to bottom
	tabsize=4,
	commentstyle=\color{mygreen},    % 设置注释颜色
	keywordstyle=\color{blue},       % 设置keyword颜色
	stringstyle=\color,     % string literal style
	%frame=single,                        % 设置有边框
	language=matlab,
}
\begin{document}
	\maketitle
	\newpage
	\setcounter{chapter}{4}
	\chapter{}
	\section{题目解答}
	\subsection{第一题:}
	\noindent 
\begin{lstlisting}
  clc;clear all;close all;
  f=-[50 100];
  A=[1 1
     2 1
     0 1];
  b=[300 400 250];
  lb=[0 0];
  ub=[inf inf];
  a=linprog(f,A,b,[],[],lb,ub)
  %answer:
  Optimization terminated.
  a =
  50.0000
  250.0000
\end{lstlisting}
\subsection{第二题答案:}
\noindent 
\begin{lstlisting}
(1)
  f=[1 2 3];
  A=[-2 1 -1
      1 1 2
     0 -1 1];
  b=[-4 8 -2];
  lb=[0 0 0 ];
  ub=[inf inf inf];
  a=linprog(f,A,b,[],[],lb,ub)
  %answer:
  Optimization terminated.

  a =

  3.0000
  2.0000
  0.0000
(2)
  f=-[5 2 4];
  A=[-3 -1 -2
  -6 -3 -5];
  b=[-2 -10];
  lb=[0 0 0];
  ub=[inf inf inf];
  a=linprog(f,A,b,[],[],lb,ub)
  answer:
  NO ANSWER

\end{lstlisting}
\subsection{第三题答案:}
\noindent \\
\textbf{解:}\par
可建立0-1规划模型,假设对于每个井位的钻与不钻两个状态设为变量$x_{i}$,其中变量的值为1时代表钻,值为0时代表不钻,由题意可得以下模型:
\begin{align}
&\notag \textnormal{min}z=\sum_{i=1}^{10}c_{i}x_{i}\\
&s.t.
\begin{cases}
x_{1}=x_{7}\\
0.5*(x_{1}+x_{7})+x_{5}=1\\
x_{3}+x_{5}=1\\
x_{4}+x_{5}=1\\
0\leq x_{i}\leq 1,\text{且}x_{1},x_{2},\dots,x_{10},\text{全部为整数}
\end{cases}
\end{align}
\subsection{第四题答案:}
\noindent 
\begin{lstlisting}
  clear all; clc;close all;
  f=-[15 10 12 10 12 11 12 9 9 9 10 20 ...
  15 17 13 18 17 9 9 13 7 13 10 13 12];
  intcon=1:25;
  A=[];
  b=[];
  Aeq=[ones(1,5),zeros(1,20);
  zeros(1,5),ones(1,5),zeros(1,15);
  zeros(1,10),ones(1,5),zeros(1,10);
  zeros(1,15),ones(1,5),zeros(1,5);
  zeros(1,20),ones(1,5);
  full(sparse(ones(1,5),[1,6,11,16,21],ones(1,5))),zeros(1,4);
  full(sparse(ones(1,5),[2,7,12,17,22],ones(1,5))),zeros(1,3);
  full(sparse(ones(1,5),[3,8,13,18,23],ones(1,5))),zeros(1,2);
  full(sparse(ones(1,5),[4,9,14,19,24],ones(1,5))),zeros(1,1);
  full(sparse(ones(1,5),[5,10,15,20,25],ones(1,5)))];
  beq=ones(10,1);
  lb=zeros(25,1);
  ub=ones(25,1);
  [a,z]=intlinprog(f,intcon,A,b,Aeq,beq,lb,ub);
  reshape(a,[5,5])
  z
\end{lstlisting}
\subsection{第五题答案:}
\noindent 
\begin{lstlisting}
  fun = @(x)50*x(1)+0.2*x(1)^2+50*x(2)+0.2*x(2)^2+...
  50*x(3)+0.2*x(3)^2+(x(1)-40)*4+(x(1)+x(2)-100)*4;
  x0=[1,1,1];
  A=[-1 0 0
      1 0 0
     -1 -1 0];
  b=[-40
     100
     -100];
  Aeq=[1,1,1];
  beq=[180];
  ub=[0 0 0];
  lb=[100 100 100];
  [x,fval]=fmincon(fun,x0,A,b,Aeq,beq,ub,lb)
  %answer:
  x =
      50.0000   60.0000   70.0001
\end{lstlisting}
\subsection{第六题答案:}
\noindent 
\begin{lstlisting}
(1)
  %build a m file:
  function [c,ceq]=answer61(x)
  c=-(1-x(1))^3+x(2);
  ceq=[];
  %code:
  fun =@(x)-x(1);
  A = [];
  b = [];
  Aeq = [];
  beq = [];
  lb = [0,0];
  ub = [inf,inf];
  nonlcon = @answer61;
  x0 = [0,0];
  x = fmincon(fun,x0,A,b,Aeq,beq,lb,ub,nonlcon)
  %answer:
  x =
  0.9988    0.0000
(2)
  same as (1)
(3)
  fun=@(x)(x(1)-3)^2+(x(2)-3)^2;
  x0=[0,0];
  A=[1,1];
  b=4;
  Aeq=[];
  beq=[];
  ub=[0,0];
  lb=[inf,inf];
  [x,fval]=fmincon(fun,x0,A,b,Aeq,beq,ub,lb)
  %answer:
  x =
  2.0000    2.0000
\end{lstlisting}
\subsection{第七题答案:}
\noindent 
\begin{lstlisting}
  f=-[15,18,21,24;
  19,23,22,18;
  26,18,16,19;
  19,21,23,17;];
  intcon=1:16;
  A=[];
  b=[];
  Aeq=[ones(1,4),zeros(1,12);
  zeros(1,4),ones(1,4),zeros(1,8);
  zeros(1,8),ones(1,4),zeros(1,4);
  zeros(1,12),ones(1,4),
  full(sparse(ones(1,4),[1,5,9,13],ones(1,4))),zeros(1,3);
  full(sparse(ones(1,4),[2,6,10,14],ones(1,4))),zeros(1,2);
  full(sparse(ones(1,4),[3,7,11,15],ones(1,4))),zeros(1,1);
  full(sparse(ones(1,4),[4,8,12,16],ones(1,4)));];
  beq=ones(8,1);
  lb=zeros(16,1);
  ub=ones(16,1);
  [a,z]=intlinprog(f,intcon,A,b,Aeq,beq,lb,ub);
  reshape(a,[4,4])
  z
  %answer:
  ans =
  0     0     0     1
  0     1     0     0
  1     0     0     0
  0     0     1     0
  z =
  -96
\end{lstlisting}
\subsection{第八题:}
\noindent 暂时空着\par
\subsection{第九题:}
\noindent \\
\textbf{解:}\par
设播放音乐节目的时间为$x_{1}$,播放新闻所用的时间为$x_{2}$,商业节目的时间为
$x_{3}$,依据题意建立线性规划模型:
\begin{align}
&\notag \textnormal{max}z=-17.5*x_{1}-40*x_{2}+250*x_{3},\\
&s.t.
\begin{cases}
x_{1}+x_{2}+x_{3}=12\\
0\leq x_{3}\leq 2.4\\
1\leq x_{2}\leq 12\\
0\leq x_{1}\leq 12
\end{cases}
\end{align}
然后发现这是一个很简单的线性规划模型,代码如下:\par
\begin{lstlisting}
  f=-[-17.5 -40 250];
  A=[];
  b=[];
  Aeq=[1 1 1];
  beq=12;
  lb=[0 1 0];
  ub=[12 12 2.4];
  [a,z]=linprog(f,A,b,Aeq,beq,lb,ub)
  %answer:
  Optimization terminated.
  a =
  8.6000
  1.0000
  2.4000
  z =
  -409.5000
\end{lstlisting}
\setcounter{chapter}{6}
\chapter{}
\section{代码部分}
	\noindent 由于本节代码较多,现将所有代码以函数方式给出,待到解题的时候直接调用函数就可以了:\par
	\subsection{图论代码}
	\noindent Dijkstra算法的Matlab函数:
\begin{lstlisting}
  function [d Q] = shorta(T)
  pp(1:length(T)) = 0; pp(1) = 1; Q = 1;
  M = max(T(:)); d(1:length(T)) = M; d(1) = 0; K = 1;
  while sum(pp)<length(T)
    tt = find(pp==0); % 找出未标记的点
    d(tt) = min(d(tt), d(K)+T(K,tt));
    ttt = find(d(tt)==min(d(tt)));
    K = tt(ttt(1)); pp(K) = 1; Q = [Q, K];
  end
\end{lstlisting}
\noindent Floyd算法的Matlab函数:
\begin{lstlisting}
  function [P, u] = f_path(W)
  % W 表示权值矩阵; P 表示最短路; % u 表示最短路的权和
  n = length(W); U = W; k = 1; % Step1 初始化
  % Step2
  while k<=n
    for i=1:n
      for j=1:n
        if U(i, j) > U(i, k) + U(k, j)
          U(i, j) = U(i, k) + U(k, j);
        end;
      end; 
    end
  k = k+1;
  end
  u = U(1, n);
  % 输出最短路的顶点
  P1 = zeros(1,n); k = 1; P1(k) = n; V = ones(1,n)*inf; kk = n;
  while kk~=1
    for i=1:n
      V(1, i) = U(1, kk) - W(i, kk);
      if V(1, i)==U(1, i)
        P1(k+1) = i; kk = i; k = k+1;
      end; 
    end;
  end
  k = 1; wrow = find(P1~=0);
  for j=length(wrow) : (-1) : 1
    P(k) = P1(wrow(j)); k = k+1;
  end
\end{lstlisting}
0-1规划模型算法:
\begin{lstlisting}
  function y=op01(W)
  %0- -1 规划模型的MATLAB 程序
  n = length(W); 
  A = zeros(n, n*n);
  intcon=1:n*n;
  for i = 1:n
    e1 = zeros(1, n);
    e1(i) = 1;
    e2 = -1*ones(1, n);
    e2(i) = 0;
    A(i, :) = repmat(e1, 1, n);
    A(i, (i-1)*n+1:i*n) = e2;
  end
  b = zeros(n, 1);
  b(1) = 1;
  b(end) = -1;
  lb=zeros(n*n,1);
  ub=ones(n*n,1);
  x = intlinprog(W,intcon,[],[],A,b,lb,ub);
  y = reshape(x, n, n);
\end{lstlisting}
\subsection{网络流模型代码}
\subsubsection{最大流模型代码}
\noindent Ford—Fulkerson 算法代码:
\begin{lstlisting}
  function f=ford(u,f)
  %Ford—Fulkerson 算法的Matlab
  n = length(u); list = [ ]; maxf = zeros(1:n); maxf(n) = 1;
  M=1000;
  while maxf(n)>0
    maxf = zeros(1, n); pred=zeros(1, n);
    list = 1; record = list; maxf(1) = M;
    while (~isempty(list)) & (maxf(n)==0)
      flag = list(1); list(1) = []; index1 = (find(u(flag, :)~=0));
      label1 = index1(find(u(flag, index1) - f(flag, index1)~=0));
      label1 = setdiff(label1, record); list = union(list, label1);
      pred(label1(find(pred(label1)==0))) = flag;
      maxf(label1) = min(maxf(flag), u(flag, label1) - f(flag, label1));
      record = union(record, label1); label2 = find(f(:, flag)~=0);
      label2 = label2'; label2 = setdiff(label2,record);
      list = union(list, label2);
      pred(label2(find(pred(label2)==0))) = -flag;
      maxf(label2) = min(maxf(flag), f(label2, flag));
      record = union(record, label2);
    end
    if maxf(n)>0
      v2 = n; v1 = pred(v2);
      while v2~=1
        if v1>0
          f(v1,v2) = f(v1, v2)+maxf(n);
        else
          v1 = abs(v1); f(v2, v1) = f(v2, v1)-maxf(n);
        end
        v2 = v1; v1 = pred(v2);
      end; 
    end; 
  end
  f; % 最后的f为最大流量矩阵
\end{lstlisting}
规划模型的代码:
\begin{lstlisting}
  function x=op02(u)
  n =length(u);
  e = [1, zeros(1, n-1)]; c = repmat(-e, 1, n);
  A = repmat(e, 1, n); A(end-n+1:end) = A(end-n+1:end) - 1;
  for i = 2:n-1
    e1 = zeros(1, n); e1(i) = 1; e2 = -1*ones(1, n); e2(i) = 0;
    A(i,:) = repmat(e1, 1, n); A(i,(i-1)*n+1:i*n) = e2; 
  end
  b = zeros(n-1,1);
  intcon=1:36;
  [x, f ] = intlinprog(c,intcon, [ ], [ ], A, b, zeros(n*n, 1), u(:));
  x = reshape(x, n, n); % 最后的f
\end{lstlisting}
\subsubsection{最小费用最大流模型}
\noindent 最小费用最大流模型Ford算法代码:
\begin{lstlisting}
  function [f,wf,zwf]=ford02(C,b)
  %最小费用最大流问题的Matlab 代码
  %C是弧容量
  %b是费用
  n = length(C);
  wf = 0; wf0 = Inf; % wf 表示最大流量, wf0 表示预定的流量值
  f = zeros(n,n); % 取初始可行流f 为零流
  while 1
    for i=1:n
      for j=1:n
        if (j~=i)
          a(i,j) = inf;
        end;
      end;
    end % 构造有向赋权图
    for i=1:n
      for j=1:n
        if (C(i,j)>0 & f(i,j)==0)
          a(i,j) = b(i,j);
        elseif (C(i,j)>0 & f(i,j)==C(i,j))
          a(j,i) = -b(i,j);
        elseif (C(i,j)>0)
          a(i,j) = b(i,j); a(j,i) = -b(i,j);
        end
      end
    end
    for i=2:n
      p(i) = inf; s(i) = i;
    end % 用Ford 算法求最短路, 赋初值
    for (k=1:n)
      pd = 1; % 求有向赋权图中vs 到vt 的最短路
      for (i=2:n)
        for (j=1:n)
          if (p(i)>p(j)+a(j,i))
            p(i) = p(j)+a(j,i); s(i) = j; pd = 0;
          end; 
        end; 
      end
      if (pd) 
        break;
      end; 
    end % 求最短路的Ford 算法结束
    if (p(n)==inf)
      break;
    end % 不存在vs 到vt 的最短路, 算法终止. 注意在求最小费
    % 用最大流时构造有向赋权图中不含负权回路, 故不出现k=n
    dvt = inf; t=n; % 进入调整过程, dvt 表示调整量
    while (1) % 计算调整量
      if (a(s(t), t)>0)
        dvtt = C(s(t), t)-f(s(t), t); % 前向弧调整量
      elseif (a(s(t), t)<0)
        dvtt = f(t, s(t)); % 后向弧调整量
      end
      if (dvt>dvtt)
        dvt = dvtt;
      end
      if (s(t)==1)
        break;
      end % 当t 的标号为vs 时, 终止计算调整量
      t = s(t);
      end % 继续调整前一段弧上的流f
      pd = 0;
      if (wf+dvt>=wf0)
        dvt = wf0-wf; pd = 1;
      end % 如果最大流量大于或等于预定的流量值
      t = n;
    while (1) % 调整过程
      if (a(s(t), t)>0)
        f(s(t), t) = f(s(t), t)+dvt; % 前向弧调整
      elseif (a(s(t), t)<0)
        f(t,s(t)) = f(t,s(t))-dvt; % 后向弧调整
      end
      if (s(t)==1)
        break;
      end % 当t 的标号为vs 时, 终止调整过程
      t = s(t);
    end
    if (pd)
      break;
    end % 如果最大流量达到预定的流量值
    wf = 0;
    for (j=1:n)
      wf = wf+f(1, j);
    end;
  end % 计算最大流量
  zwf = 0;
  for (i=1:n)
    for (j=1:n)
      zwf = zwf+b(i, j)*f(i, j);
    end
  end % 计算最小费用
\end{lstlisting}
最小费用最大流规划算法代码:
\begin{lstlisting}
  function [f,wf]=op03(C,w)
  n = length(C);
  e = [1, zeros(1, n-1)]; c = repmat(-e, 1, n);
  A = repmat(e, 1, n); A(end-n+1:end) = A(end-n+1:end) - 1;
  for i = 2:n-1
    e1 = zeros(1, n); e1(i) = 1; e2 = -1*ones(1, n); e2(i) = 0;
  A(i,:) = repmat(e1, 1, n); A(i,(i-1)*n+1:i*n) = e2; 
  end
  b = zeros(n-1,1);
  intcon=1:n*n;
  [x, fv ] = intlinprog(c, intcon,[ ], [ ], A, b, zeros(n*n, 1), C(:));
  f = reshape(x, n, n);
  A = repmat(e, 1, n);
  for i = 2:n
    e1 = zeros(1, n); e1(i) = 1; e2 = -1*ones(1, n); e2(i) = 0;
    A(i,:) = repmat(e1, 1, n); A(i,(i-1)*n+1:i*n) = e2; 
  end
  b = [-fv; zeros(n-2,1); fv ];
  [x, gv ] = linprog(w, [ ], [ ], A, b, zeros(n*n, 1), C(:));
  wf = reshape(x, n, n); % 最小费用最大流量矩阵
\end{lstlisting}
\subsection{最优连线模型与最优环游模型代码}
\subsubsection{最小生成树代码}
\noindent 避圈法代码:
\begin{lstlisting}
  function A = avoidcircle(W)
  [m, n] = size(W);
  e = 0;
  for i = 1 : n
    for j = i : n
      if W(i, j) ~= 0
        e = e + 1;
        E(e, :) = [i, j, W(i, j)];
      end
    end
  end
  % 按权值大小排列边的顺序
  for i = 1 : e - 1
    for j = i + 1 : e
      if E(i, 3) > E(j, 3)
        temp = E(j, :);
        E(j, :) = E(i, :);
        E(i, :) = temp;
      end
    end
  end
  A = zeros(1, 3); S = 1 : n;
  for i = 1 : e
    if S(E(i, 1)) ~= S(E(i, 2))
      A = cat(1, A, E(i,:));
      indicator = S(E(i, 1));
      for j = 1 : n
        if S(j) == indicator
          S(j) = S(E(i, 2));
        end
      end
    end
  end
  A(1, :) = [];
\end{lstlisting}
破圈法代码:\\
暂时空着:
\begin{lstlisting}
\end{lstlisting}
\subsubsection{最优环游模型:}
\noindent 改良圈算法代码:
\begin{lstlisting}
  function [circle,sum]=circle1(a)
  a = a+a';
  c1 = [5 1:4 6];
  L = length(c1);
  flag = 1;
  while flag>0
    flag = 0;
    for m=1:L-3
      for n=m+2:L-1
        if a(c1(m),c1(n))+a(c1(m+1),c1(n+1))< a(c1(m),c1(m+1))+a(c1(n),c1(n+1))
          flag = 1;
          c1(m+1:n) = c1(n:-1:m+1);
        end; 
      end; 
    end; 
  end
  sum1 = 0;
  for i=1:L-1
    sum1 = sum1+a(c1(i),c1(i+1));
  end
  circle = c1;
  sum = sum1;
  c1 = [5 6 1:4]; % 改变初始圈,最后一个顶点不动
  sum1 = 0; flag = 1;
  while flag>0
    flag=0;
    for m=1:L-3
      for n=m+2:L-1
        if a(c1(m),c1(n))+a(c1(m+1),c1(n+1)) < ...
          a(c1(m),c1(m+1))+a(c1(n),c1(n+1))
          flag=1; c1(m+1:n)=c1(n:-1:m+1);
        end; 
      end; 
    end; 
  end
  sum1 = 0;
  for i=1:L-1
    sum1 = sum1+a(c1(i),c1(i+1));
  end
  if sum1<sum
    sum = sum1;
    circle = c1;
  end
\end{lstlisting}
规划算法代码:
\begin{lstlisting}
%此为错误代码,待修正
  function x=op04(a)
  n = length(a); a = a+a';
  A = kron(eye(n), ones(1, n));
  A(n+1:2*n, :) = repmat(eye(n), 1, n);
  b = ones(2*n, 1);
  intcon=1:36;
  [x, f ] = intlinprog(a(:),intcon, [ ], [ ], A, b,zeros(36,1),ones(36,1));
  x = reshape(x, n, n);
\end{lstlisting}
\section{题目解答:}
\subsection{第一题答案:}
建立图论矩阵:
\[
\mathbf{A} = \left[
\begin{array}{ccccccc}
0&1&4&0&0&0&0\\
0&0&5&3&5&0&0\\
0&0&0&0&0&2&0\\
0&0&0&0&5&7&3\\
0&0&0&0&0&0&7\\
0&0&0&0&0&0&2\\
\end{array} \right]
\]
然后直接作为矩阵带入函数即可,但是需要注意的是带入时将零换成极大值即可,求解代码:
\begin{lstlisting}
%(1)Dijkstra算法:
  m=10000;
  W=[0,1,4,m,m,m,m;
     m,0,5,3,5,m,m;
     m,m,0,m,m,2,m;
     m,m,m,0,5,7,3;
     m,m,m,m,0,m,7;
     m,m,m,m,m,0,2;
     m,m,m,m,m,m,0;];
  [d,Q]=shorta(W)
  %answer
  d =
     0 1 4 4 6 6 7
  Q =
     1 2 3 4 5 6 7
%(2)Floyd算法:
  [d,Q]=f_path(W)
  %answer
  d =
    1     2     4     7
  Q =
    7
%(3)规划算法:
  d=op01(W)
  %answer
  LP:Optimal objective value is 7.000000.  
  d =
  0 1 0 0 0 0 0
  0 0 0 1 0 0 0
  0 0 0 0 0 0 0
  0 0 0 0 0 0 1
  0 0 0 0 0 0 0
  0 0 0 0 0 0 0
  0 0 0 0 0 0 0
\end{lstlisting}
\subsection{第二题答案:}
\noindent 解法同上,求解代码:
\begin{lstlisting}
%(1)Dijkstra算法:
m=10000;
W=[0,9,8,m,m,m,m;
   m,0,5,2,1,m,m;
   m,m,0,8,m,7,m;
   m,m,m,0,2,3,m;
   m,m,m,m,0,m,3;
   m,m,m,m,m,0,4;
   m,m,m,m,m,m,0;];
   [d,Q]=shorta(W)
   %answer
   d =
     0 9 8 11 10 14 13
   Q =
     1 3 2 5 4 7 6
%(2)Floyd算法:
  [d,Q]=f_path(W)
  %answer
  d =
    1 2 5 7
  Q =
    7
%(3)规划算法:
  d=op01(W)
  %answer
  LP:Optimal objective value is 13.000000.  
  d =
     0 1 0 0 0 0 0
     0 0 0 1 0 0 0
     0 0 0 0 0 0 0
     0 0 0 0 0 0 0
     0 0 0 0 0 0 1
     0 0 0 0 0 0 0
     0 0 0 0 0 0 0
\end{lstlisting}
\subsection{第三题答案:}
\noindent 求解代码:
\begin{lstlisting}
  f=zeros(6,6);
  u=[0,16,20,0,0,0,;
  0,0,0,10,0,10;
  0,0,0,6,6,0;
  0,0,0,0,0,10;
  0,0,0,0,0,16;
  0,0,0,0,0,0;];
  f=ford(u,f)
  %answer:
  f =
    0 16 10 0 0 0 
    0 0 0 6 0 10
    0 0 0 4 6 0
    0 0 0 0 0 10
    0 0 0 0 0 6
    0 0 0 0 0 0
\end{lstlisting}
\subsection{第四题答案:}
\noindent 求解代码:
\begin{lstlisting}
  f=zeros(7,7);
  u=[0,7,8,6,0,0,0;
     0,0,0,0,5,0,0;
     0,3,0,2,5,3,0;
     0,0,0,0,0,10;
     0,0,0,0,3,0,9;
     0,0,0,0,0,0;];
  f=ford(u,f)
  %answer:
  f =
    0 5 8 5 0 0 0 
    0 0 0 0 5 0 0
    0 0 0 0 5 3 0
    0 0 0 0 0 5 0
    0 0 0 0 0 0 10
    0 0 0 0 0 0 8
    0 0 0 0 0 0 0 
\end{lstlisting}
\subsection{第五题答案:}
\noindent 求解代码:
\begin{lstlisting}
  f=-[2,3,4,1,7;
  3,4,2,5,6;
  2,5,3,4,1;
  5,2,3,2,5;
  3,7,6,2,4];
  intcon=1:25;
  A=[];
  b=[];
  Aeq=[ones(1,5),zeros(1,20);
  zeros(1,5),ones(1,5),zeros(1,15);
  zeros(1,10),ones(1,5),zeros(1,10);
  zeros(1,15),ones(1,5),zeros(1,5);
  zeros(1,20),ones(1,5);
  full(sparse(ones(1,5),[1,6,11,16,21],ones(1,5))),zeros(1,4);
  full(sparse(ones(1,5),[2,7,12,17,22],ones(1,5))),zeros(1,3);
  full(sparse(ones(1,5),[3,8,13,18,23],ones(1,5))),zeros(1,2);
  full(sparse(ones(1,5),[4,9,14,19,24],ones(1,5))),zeros(1,1);
  full(sparse(ones(1,5),[5,10,15,20,25],ones(1,5)))];
  beq=ones(10,1);
  lb=zeros(25,1);
  ub=ones(25,1);
  [a,z]=intlinprog(f,intcon,A,b,Aeq,beq,lb,ub);
  a=reshape(a,[5,5])
  %answer:
  a=
    0 0 0 0 1
    0 0 0 1 0
    0 1 0 0 0
    1 0 0 0 0
    0 0 1 0 0
\end{lstlisting}
\subsection{第六题答案:}
\noindent 求解代码:
\begin{lstlisting}
  C = [0,6,2,1,0;0,0,0,0,0;0,2,0,10,3;0,4,0,0,0;0,0,0,0,0]; % 弧容量
  b = [0,5,9,4,0;0,0,0,0,0;0,3,0,4,2;0,3,0,0,0;0,0,0,0,0];
  [f,wf,zwf]=ford02(C,b)
  %answer:
  f =
    0     0     2     0     0
    0     0     0     0     0
    0     0     0     0     2
    0     0     0     0     0
    0     0     0     0     0
  wf =
    2
  zwf =
    22
\end{lstlisting}
\subsection{第七题答案:}
\noindent 求解代码:
\begin{lstlisting}
  C = [0,2,8,0,0,0;
       0,0,5,2,0,0;
       0,0,0,0,3,0;
       0,0,1,0,0,6;
       0,0,0,4,0,7;
      0,0,0,0,0,0;]; % 弧容量
  b = [0,8,7,0,0,0;
       0,0,5,9,0,0;
       0,0,0,0,9,0;
       0,0,2,0,0,5;
       0,0,0,6,0,10;
       0,0,0,0,0,0];
  [f,wf,zwf]=ford02(C,b)
  %answer:
  f =
    0     2     3     0     0     0
    0     0     0     2     0     0
    0     0     0     0     3     0
    0     0     0     0     0     2
    0     0     0     0     0     3
    0     0     0     0     0     0
  wf =
  5
  zwf =
  122
\end{lstlisting}
\subsection{第八题答案:}
\noindent 求解代码:
\begin{lstlisting}
  a(1, 1:6) = [0,3, 7, 4, 0, 0];
  a(2, 1:6) = [3, 0, 2,0,9,0];
  a(3, 1:6) = [7,2,0,1,6,3];
  a(4, 1:6) = [4,0,1,0,0,4];
  a(5, 1:6) = [0,9,6,0,0,3]; a(6, :)=0;
  aviodcircle(a)
  %answer:
  ans =
       3     4     1
       2     3     2
       1     2     3
       3     6     3
       5     6     3
\end{lstlisting}
\subsection{第九题答案:}
\noindent 求解代码:
\begin{lstlisting}
  a(1,1:9)=[0,2,1,3,0,0,0,0,0];
  a(2,1:9)=[2,0,4,0,5,6,0,0,0];
  a(3,1:9)=[1,4,0,3,5,0,0,0,0];
  a(4,1:9)=[3,0,5,0,6,0,0,8,0];
  a(5,1:9)=[0,5,3,6,0,4,0,0,0];
  a(6,1:9)=[0,2,0,0,4,0,5,0,3];
  a(7,1:9)=[0,0,0,0,3,5,0,4,1];
  a(8,1:9)=[0,0,0,8,7,0,4,0,2];
  a(9,1:9)=[0,0,0,0,0,0,0,0,0];
  avoidcircle(W)
  %answer:
  ans =

       2     5     1
       2     4     2
       4     6     3
       5     7     3
       2     3     5
       1     3     8
\end{lstlisting}
\subsection{第十题答案:}
\noindent 求解代码:
\begin{lstlisting}
  a(1,2)=10;a(1,3)=20;a(1,4)=30;a(1,5)=40;a(1,6)=50;
  a(2,3)=18;a(2,4)=30;a(2,5)=25;a(2,6)=21;
  a(3,4)=5;a(3,5)=10;a(3,6)=15;
  a(4,5)=8;a(4,6)=16;
  a(5,6)=18;
  a(6,:)=0;
  [circle,sum]=circle1(a)
  %answer:
  circle =
  5     4     3     1     2     6
  sum =
  64
\end{lstlisting}

\end{document}