\documentclass[a4paper,UTF8,cs4size]{ctexart}
\usepackage{xeCJK}
\usepackage{fancyhdr}
\pagestyle{fancy}
\fancyhf{}
\setlength{\headheight}{14pt}
\chead{\songti\zihao{-5}计算机科学与技术学院管理信息系统期末论文}
\cfoot{\thepage}
\title{\heiti\zihao{2}\bfseries 管理信息系统及其应用}
\author{\zihao{3}\kaishu 钱昌发\\\zihao{3}\kaishu 学号:A41714036\\\zihao{3}\kaishu 应用统计学}
\date{\songti 2019年6月15日}
\setCJKmainfont{宋体}
\setmainfont{Times New Roman}
\linespread{1.5}
\ctexset{
section = {
name = {,、},
number = \chinese{section},
format= {\heiti}
},
subsection={
	name={(,)},
	number=\chinese{subsection},
	format={\songti}
},
bibname ={\heiti\zihao{4}参考文献},
abstractname={\heiti 摘要}
}

\begin{document}
	\maketitle
	\newpage
	\tableofcontents
	\newpage
	\begin{abstract}
		管理信息系统(Management Information Systems简称MIS)、是一个不断发展的新型学科,MIS的定义随着计算机技术和通讯技术的进步也在不断更新,在现阶段普遍认为管理信息系统MIS、是由人和计算机设备或其他信息处理手段、组成并用于管理信息的系统。\cite{q4}
	\end{abstract}
	\section{信息管理系统的发展及历史}
	自人类进入信息时代开始,数据就开始变得繁杂而又庞大起来,这样一来如何从繁杂而又庞大的数据提取到对自己有利的数据就一直是困扰人的问题。\par
	\subsection{古代信息传播与管理}
	起初的时候人类通过人们通过驿站、飞鸽传书、烽火报警、符号、语言、眼神、触碰等方式进行信息传递。这些信息所具有的特点都是简单但是低效,一般来说不会有专门有人来管理信息,但是都会有专人或者物来传播信息,比如驿站的管理人员和邮递员之类的职业,由于信息的简单和低效,所以这些人员大多数时候都兼职管理信息,比如驿站的管理人员会将信息分为加急或者不加急的信件,邮递员会将信件分门别类,将同一地区信件放在一起。\par
	起初网络未被发明的时候或者网络未发展完善的时候,这样的简单信息管理系统已经足够维持信息的最高效的运转,但是有关信息的整理技术已经开始在20世纪初的时候出现,比较经典的就是统计学的飞速发展。\par
	\subsection{信息系统的雏形与统计学的出现}
	统计学的产生与统计实践活动的发展是密不可分的,统计作为一种社会实践活动,是随着记数活动面产生和发展起来的统计发展的历史可以追刚到远古的原始社会。在那时,人们按都落居住在起,打猎、 捕鱼后就委算算有多少人、多少食物,以便进行分配,这是统计的雏形。我国夏禹时代就有了人口数据和土地数据的记载,这说明在夏朝就已经有了统计的萌芽。为了赋税.徭役和兵役的需要,世界各国历代都有田亩和户口的记录,在奴隶社会和封建社会主要是对人口、土地和财产进行统计。\par
	13世纪至16世纪中叶,欧洲各国资本主义出现不平衡发展,各国的国情也不一致,欧洲各主要国家都深感有调查国内外情况的必要,从意大利开始,各国相继进行本国和他国的历史沿革、地理条件、国家典章制度、财政收入、军事实力、居民风俗习惯、国家工商业交通运输等国情调查。当各国调查资料积累到一定数量时,开始有学者将之汇编成册,并开始初步研究。德国学者芒斯特所编的《世界志》,是第一部反映这些国家财富调查的科学统计著作。随后意大利、英国等国也都有学者编著了有关世界各国或本国国情方面的著作。这些著作方面为当政者或者准备从政者提供必要的国内外知识,另一方面也为国家制定方针政策以及资本家在国内外经商、争夺海外市场和开拓殖民地提供定的国内外知识。这些著作主要用简单的文字记述方法来论述和反映国情方面的知识,缺少必要的比较因果等分析方法后人称之为“国势论"。\cite{q1}\par
	很显然“国势论”的发展历程与信息管理密不可分的,可以看出逐渐增加的各类信息已经逐渐开始影响人们的判断,人们已经需要建立一种管理信息的方式,而这个国势论就是比较经典管理信息系统在统计学方面的发展,而管理信息系统在许多方面都有应用,由于专业限制,这里不再赘述。\par
	\section{大数据时代与信息管理系统}
	\subsection{网络与大数据时代}
	随着网络的发展,通过因特网的到来的信息开始变得传播快速而且高效,这时候大数据时代就开始到来。大数据,又称为巨量资料,指的是在传统数据处理应用软件不足以处理的大或复杂的数据集的术语。大,也可以定义为来自各种来源的大量非结构化或结构化数据。从学术角度而言,大数据的出现促成了广泛主题的新颖研究。这也导致了各种大数据统计方法的发展。大数据并没有统计学的抽样方法;它只是观察和追踪发生的事情。因此,大数据通常包含的数据大小超出了传统软件在可接受的时间内处理的能力。由于近期的技术进步,发布新数据的便捷性以及全球大多数政府对高透明度的要求,大数据分析在现代研究中越来越突出。在这其中能对大数据进行管理更加重要。\cite{q2}\par
	\subsection{大数据在信息管理系统中的具体功能}
	大数据信息处理系统是一个适应时代发展的新词汇,云计算、物联网、互联网 + 以及智慧城市等各种依靠于大数据发展的新兴技术,对人们的传统观念进行不断冲击,使人们的传统观念发生了或多或少的改变,利用大数据信息处理技术有利于对各种数据进行处理和分析,可对手机进行定位,获得购物车在商场的停留时间,优化商场的布局,从而达到一定的管理效果,利用大数据信息处理技术有利于改善管理模式。\par
	数据资产是企业非常重要的一个资产,大数据信息技术有利于对企业生产与发展的各个方面信息进行集中管理,然后通过信息共享,把信息传播出去,加快企业与其他企业在信息上的传递与交流,有利于企业及时对内部结构与环境进行调整与升级。\par
	利用大数据处理技术对信息技术进行处理和存储也是信息技术处理的一个重要手段,大数据对于信息的存储量更大,在信息存储的同时也采取了对信息进行加密手段,提高数据的保密性和数据的完整性。\par
	通过大数据技术中的数据挖掘技术以及统计建模等其他技术可对数据进行一定模拟,寻找到一定规律,了解一些数据处理或者信息方面存在的风险,然后及时采取措施达到规避风险或降低风险损失的目的。利用大数据可对一些数据的传输进行了解,然后监控存在风险的状态,及时向相关人员推送风险信息,引起大家对于存在风险的关注。\par
	提供一些实时的数据流,能给智能审计提供了一定便利,传统的审计方式可保证业务的真实性,但是基于大数据信息处理技术不仅能保证信息的真实性,同时也能最大程度上消除信息欺诈和舞弊的可能性。
	管理信息系统在现代企业管理中的地位是不可替代的,这以信息为核心系统性的工程能够有效地提升企业组织的运作效率,在企业组织核心竞争力的塑造\par
	\section{大数据与信息管理系统在企业中的作用}
	管理信息系统在现代企业管理中的地位是不可替代的,这以信息为核心系统性的工程能够有效地提升企业组织的运作效率,在企业组织核心竞争力的塑造和经济效益的提高上起着不可或缺的作用,在信息时代和知识经济时代大有可为。\par
	但是,管理信息系统的成功运用除了系统本身的作用外,还要求外部资源的配合,企业的管理体制、绩效体系和人才配置等都必须与管理信息系统相配套.通过系统性的资源整合,更好地发挥管理信息系统的作用。简而言之,管理信息系统在现代企业组织管理中,特别是在我国这样的市场经济体制还不完善,企业管理思想稍显落后的发展中国家的企业管理中还有很长的路要走。\cite{q3} \par
	
	\begin{thebibliography}{99}
		\bibitem{q4}\songti\zihao{5}{维基百科.信息管理系统[EB/OL]https://zh.wikipedia.org/wiki/管理信息系统\#参考文献}
		\bibitem{q1}\songti\zihao{5}{张春国.统计学[M].中国成都:西南财金大学出版社,2016:04-05.}
		\bibitem{q2}\songti\zihao{5}{钟耀华.大数据环境下的管理信息系统发展研究[J].信息与电脑(理论版),2019(08):124-125.}
		\bibitem{q3}\songti\zihao{5}{郑洋洋.浅谈信息管理系统在现代企业管理中的作用[EB/OL]https://wenku.baidu.com/view/63d335ccd4d8d15abf234e1b.html}
	\end{thebibliography}
\end{document}