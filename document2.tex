\documentclass[11pt,a4paper]{ctexart}
\usepackage{fancyhdr}%设置页脚页眉需要的包
\usepackage{xeCJK}
\usepackage[T1]{fontenc}
\author{应用统计学\ 钱昌发\ A41714036} 
\title{论反对“台独”分裂势力的路径设计}
\date{2019/5/30}
\pagestyle{myheadings}
\markboth{new}{1}

\begin{document}
  \maketitle%输出标题
  \newpage%另起一页
  \tableofcontents%命令输出论文目录
  \newpage
  \begin{abstract}
  	到现在为止,经过几十年的发展“台独意识”已透过历史、文化领域进入台湾民众,特别是青少年的思想意识中,国家、民族、文化的认同观念已逐渐从“中国”、“中华”转向“台湾”。“文化台独”不仅对两岸关系造成危害,而且对台湾人民的思想意识、价值观念、价值取向、理想信念、精神追求以及国家、民族、文化的认同也造成混乱。
  \newline%另起一行

  \centering%使得关键字居中
  \textbf{关键字:}台独、中国共产党、国民党
  \end{abstract}

  \section{历史背景}
    \subsection{清朝时期} 
    台湾独立势力最早可以追溯到清朝时期,郑成功的后代郑经曰:“远绝大海,建国东宁,于版图疆域之外,别立乾坤。”这一般认为这是最早的“汉人台独思想”,但是由于当时时代背景缘故,这一思想并没有持续太久便消失了。\par
    之后便在清日甲午战争清军战败后,清朝将台湾和澎湖割让给日本。台湾人民抵抗日军的接收,于1895年5月25日在拥立当时台湾巡抚唐景崧建立“台湾民主国”。同年10月19日,刘永福总统兵败内渡;两日后台南陷落。台湾民主国仅存150天,最终被日军剿灭。台湾民主国虽有独立之名却无独立之实,有资料认为,当时在台清国官员只是在地方士绅的压力下,为了抗拒日本统治,又避免台湾的武装反抗与清国扯上关系,才提出独立建国的主张,其宣言中并主张“恭奉正朔,遥作屏籓”。虽然有部分支持台独的学者,试着将民主国视为一个“事实”独立的国家,而将其放在台湾人民从17世纪以来反抗异族统治的脉络下来解读。但真正台湾民族意识的萌芽则要到以李秉瑞为首的21年武装抗日失败后,林献堂所领导的台湾文化协会推动的“台湾民族”运动。\cite{q1}
    
     \subsection{日占时期} 
     1928年成立的“台湾共产党”(日本共产党台湾民族支部),成立时也提出“台湾民族”的概念,并明确主张建立台湾共和国,这是台湾有史以来第一个提出台独主张的政治团体。台湾民族的最初形成,就是在殖民与封建的交织过程中在进行的。之后,台湾又历经中华民国和日本帝国主义的统治,而终于在这样的历史过程中建立起台湾民族的共同利益和共同连带感。\par
     而在第二次世大战之后,中国打败日本成为第二次世界大战的战胜国,在战后的卡罗会议上提出将日本所占领的台湾等地归还给中华民国,然后随后的受降仪式上中华民国接收台湾,虽然此时台湾地区处于未定主权,但实际上已经成为国民党统治地区。
     \subsection{国民党时期}
     1949年之后台湾地区被败逃而来的蒋介石所占领成为国民党的基地,此时蒋家开始了对于台湾接近四十年的统治,他们对于台湾采取封锁策略,但是不主张独立,主张“反共复国“\par
     1988年之后蒋介石的儿子蒋经国去世,副总统李登辉依据宪法继任,而其在1980年成立的民进党在其在任期间逐渐成为台湾独立的主要推动力,199年民主进步党在《基本纲领:我们的主张》开宗明义论述:“建立主权独立自主的台湾共和国”,此基本纲领之论述常被称为“台独党纲”。党纲该条第一款末了注明“基于国民主权原理,建立主权独立自主的台湾共和国及制定新宪法的主张,应交由台湾全体住民以公民投票方式选择决定。”的程序。1999年的“台湾前途决议文”则表明“台湾主权独立,与中华人民共和国互不隶属,既是历史事实,也是现实状态。\par
     2000年中华民国总统选举,主张台湾独立的民主进步党人、前台北市市长陈水扁以39.3\% 得票率当选中华民国总统,成功实现政党轮替,主张台独、成立仅13年的民进党首次取得执政权,使台独运动进入一个新的里程,虽然他曾公开宣布“四不一没有”,但他其后曾提出许多新的方案,包含一中一台,一边一国,四阶段论等主张法理台湾独立,并主张台湾前途应由台湾人民决定,中国大陆政府强烈反对。\par
     2008年中华民国总统选举,马英九当选总统,国民党重新执政,二度政党轮替,台独势力受挫。\par
     2012年中华民国总统选举,民进党再次败于国民党,使得民进党重新审视对中国大陆的政策。5月31日,中华人民共和国国台办提出明确的说法表示:”民进党坚持“一边一国”的“台独”主张,这是他们和我们交往的最大障碍。”,使台独运动从在内部的对抗转变为在外部对抗中华人民共和国与国际压力的运动。\par
     对于台独势力中方从来都是采取零容忍的态度,江泽民总统曾经提出:“只有坚持一个中国原则,我们才有的谈”,甚至曾经的国务院总理朱镕基同志曾经说过:”谁要是搞台湾独立就没有好下场!”\cite{q2}
     

  \section{台独的影响} 
  第一,国家宪法将遭到践踏,治国安邦之本将发生动摇。“台独”严重损害中国的国家主权、尊严和领土完整,动摇国家的根本。如果听任“台独”得逞,中国的武装力量就没有起到护法卫国的责任,就是最大的违宪和失职.\par
  第二,境内外其他分裂势力必将群起效尤,我国安全稳定的政治局面有可能被破坏,一个团结、和睦、统一的中国,很可能变为四分五裂、狼烟四起的中国,国家将永无宁日。\par
  第三,我300万平方公里海洋国土的一半和大量海洋资源将随之丧失,我对外开放的主要通道即东部海上通道,从此将处在别人的战略监控之下,中国的东南沿海地区将直接暴露在外部势力的威胁之下,地缘环境将严重恶化。同样,海南岛若失去与台湾岛的彼此呼应,其保卫南部海疆的支撑作用也将降低,南沙的海权将更加难以维护,中国“南下”的战略通道在战时有可能面临被封阻的危险。\par
  第四,台湾倘若沦为某些西方大国的附庸,一些对中国怀有敌意的国家必以“保护台湾”为名,进一步加强在中国周边的军事存在,比现在更频繁地打“以台制华”牌,我整个国家的安全环境将严重恶化。\par
  第五,我国的国际声望因不能有力地维护国家主权和领土完整而严重受损,日后将难以在国际事务中有效维护自己的民族权益。\par
  第六,势必长期消耗我外交资源和牵制我以经济建设为中心的大局。台湾一旦分裂出去,我经济最发达的东南地区就将直接面临外部威胁,商场变为战场,外商很可能望而却步,投资环境将严重恶化,势必影响我“三步走”的发展战略。\par
  第七,最危险的是,假如让台湾在我们共产党手里丢掉,中国人民将对我党能否真正代表“最广大人民的利益”提出质疑,我执政党的地位就会面临严峻考验\cite{q3}
  \section{反对台独的未来形式}
  而现在为止,经过几十年的发展“台独意识”已透过历史、文化领域进入台湾民众,特别是青少年的思想意识中,国家、民族、文化的认同观念已逐渐从“中国”、“中华”转向“台湾”。“文化台独”不仅对两岸关系造成危害,而且对台湾人民的思想意识、价值观念、价值取向、理想信念、精神追求以及国家、民族、文化的认同也造成混乱。思想的混乱会带来人心的浮动,由此引起社会的动荡,可以预见,台湾的社会混乱和矛盾会不断加剧,为了转移矛盾,必然会制造两岸紧张关系,使两岸关系陷入恶性之中,在这种恶性循环中播下仇恨的种子,离间两岸人民的感情,协迫台湾人民走“台独”之路。一旦“台独意识”成为台湾的主流意识,陈水扁当局就会假借民意公开宣布“台湾独立”。“台独”即意味着战争。\par
  台湾和大陆自古就是不可分割的- . 个整体,中华民族的发展史是两岸全体中国人共同创造的历史。海峡两岸人民的中华民族认同,对于在当代发展两岸关系,促进祖国完全统一方面, 有着不可替代的重要作用。\par
  通过研究发现,由于受到台湾近代特殊的历史和当代台独势力推行“去中国化”政策的影响,台湾民众对于中华民族的认同有弱化和分离化的趋势,这是两岸关系的发展中所不乐见的现象。纵观两岸六十年的风风雨雨,不难看出,对于中华民族的认同,对两岸关系有如下几方面影响:\par
  首先,中华民族认同是两岸关系发展的基础,在两岸政策层面发挥着根本性的指导作用。在以和平和发展为时代主题的今天,认同中华民族,两岸关系就会有较大发展,反之两岸关系就止步不前。这-点在台清大陆政策上表现尤其明显。李登辉、陈水扁先后共执政二十年,为达到台独目的而否认中华民族和中华文化,两岸关系因其对抗性和分离性大陆政策而发展较慢。马英九上任后坚持“九二共识”,推动积极的大陆政策,其政策中隐性却又起根本性引导作用的就是对于中华民族的认同。两岸关系在这一共同基础上有了大步前进。\par
  其次,中华民族认同是遏制岛内台独势力的重要法宝。台独势力别有用心的鼓吹所谓的“台湾文化”、“台湾民族”,企图借助文化台独进而达到其政治目的。然而,中华民族几千年的发展历史和由此产生的中华文化,使得文化台独的立场不攻自破。热爱台湾、认同台湾的“台湾意识”是中华文化乡土意识在台湾社会的体现,这与台独意识有着根本的区别。对于中华民族的认同,可以更好的将台湾意识与台独意识区别开来,并成为抵制台独思想最根本的心理认同。\par
  最后,中华民族认同是目前两岸在“一个中国”认同存在分歧下最易达成共识的契合点。两岸对于“一个中国”有着各自的表述,在“国家认同”上存在根本性的争议。为了实现两岸关系的和平发展,将现阶段两岸关系的发展凝聚和巩固在中华民族的认同之下,在实现两岸关系良性发展的同时增进两岸同胞同属中华民族大家庭的情感和认同,从而为祖因的完全统-奠定良好的思想基础。\cite{q4}
  \begin{thebibliography}{99}
  	\bibitem{q1} 维基百科:台湾独立运动等词条
  	\bibitem{q2} 百度百科:蒋介石、国民党总统、台独等词条
  	\bibitem{q3} 李成民[\emph{湖南大学}] :民进党``台独''思想研究
  	\bibitem{q4} 陈媛[\emph{中央民族大学}]:中华民族认同与两岸关系发展探析
  	
  \end{thebibliography}
  
\end{document} 